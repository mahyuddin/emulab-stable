\documentclass{article}
\title{GEC 8 ProtoGENI Tutorial}
\date{20th July, 2010}
\author{James Griffioen \and Hussamuddin Nasir \and Robert Ricci \and
  Gary Wong}
\begin{document}
\maketitle
\section{Basic slice operation}
\begin{enumerate}
\item Connect to the client machine:\\
ssh to \emph{username}{\tt @users.emulab.net}\\
{\tt cd /proj/gec8tutorial/scripts}
\item Store your passphrase (optional):\\
{\tt ./rememberpassphrase.py}
\item Inspect your account:\\
{\tt ./showuser.py} \emph{username}
\item Register a slice:\\
{\tt ./registerslice.py -n} \emph{username}{\tt slice}
\item Inspect your account (again):\\
{\tt ./showuser.py} \emph{username}
\item Create a sliver:\\
{\tt ./allocatenodes.py} -n \emph{username}{\tt slice} {\tt gec.rspec}
\item Log in to the components:\\
{\tt ssh} \emph{client-hostname}
\item Clean up the slice:\\
{\tt ./deleteslice.py} -n \emph{username}{\tt slice}
\item Remove your passphrase (optional):\\
{\tt ./forgetpassphrase.py}
\end{enumerate}
\begin{itemize}
\item Optional extras:\\
Try editing a copy of {\tt gec.rspec}\\
{\tt http://www.protogeni.net/trac/protogeni/wiki/RSpecExamples}\\
{\tt ./listcomponents.py}\\
{\tt ./discover.py}
\end{itemize}

\section{Instrumentation tools}
Please refer to the documentation available from:
\begin{quote}{\tt http://www.netlab.uky.edu/p/instools/}\end{quote}
\noindent{}Note that the instrumentation tools are available on {\tt users.emulab.net}, in:
\begin{quote}{\tt /proj/gec8tutorial/tarfiles/INSTOOLSv3\_0.tgz}\end{quote}
\begin{enumerate}
\item To create an experiment:\\
{\tt python createExp.py -n} \emph{slicename} {\tt $\backslash$}\\
{\tt -m https://www.uky.emulab.net/protogeni/xmlrpc/cm
rspecs/gec8-kentucky.rspec}
\item To instrument this experiment:\\
{\tt python instrumentize.py -n} \emph{slicename}
\item To generate traffic from {\tt ky1} to {\tt ky2}:\\
{\tt /usr/bin/traffic\_gen.sh} (on {\tt ky1})
\end{enumerate}

\section{Emulab frontend}
This part of the tutorial is described on the WWW:
\begin{quote}{\tt http://www.protogeni.net/trac/protogeni/wiki/GEC8Tutorial}\end{quote}
\end{document}
