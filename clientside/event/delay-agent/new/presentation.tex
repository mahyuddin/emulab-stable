\documentclass[12pt]{article}

\title{Emulab Delay Agent Presentation}			% used by \maketitle
\author{Jon Duerig and Ryan Jackson }		% used by \maketitle
\date{April 18, 2008}					% used by \maketitle

\begin{document}
\maketitle						% automatic title!
\section{Emulab Delay Agent}

\subsection{Purpose of delay agent}
\begin{itemize}
\item
Control link properties: bandwidth, delay, packet loss, etc.
\end{itemize}

\subsection{Problems with existing implementation}
\begin{itemize}
\item
Dependent on FreeBSD's DummyNet
\end{itemize}

\subsection{Goals of New Implementation}
\begin{itemize}
\item
Support link up/down, bandwith limits, and delay
\item
Support Linux in end-node shaping configuration
\item
Support Linux as delay node
\item
Clean API between OS-independent front-end and OS-dependent back-end
\end{itemize}

\section{New Delay Agent Capabilities}

\begin{itemize}
\item
New delay agent can currently bring link up or down, modify
bandwidth limits, and change packet delay.
\begin{itemize}
\item
Modifications to packet loss rate are handled by front-end, but
not yet passed to back-end.  However, back-end uses the kernel's
packet loss code to implement link up/down.
\end{itemize}
\end{itemize}

\begin{itemize}
\item
Support for joth FreeBSD and Linux
\begin{itemize}
\item
Runs in both delay node and end node shaping configurations on FreeBSD.
\item
Runs in end node shaping configuration on Linux.  Delay node configuration not tested, but there is no reason it should not work.
\end{itemize}
\end{itemize}

\section{Design}

\subsection{Agent has two main parts}
\begin{itemize}
\item
OS-independent front-end
\item
OS-dependent back-end
\end{itemize}

\section{Front-End}

\begin{itemize}
\item
Front end receives events from the Emulab event system
\begin{itemize}
\item
Events are used to modify link parameters or bring it up/down.

\item
Event includes information about which link or pipe to modify.

\end{itemize}
\item
The event is parsed and a Parameter object is constructed.
\item
The front end uses the link or pipe information in the event to look up the correct back-end pipe object.
\item
Parameter object is passed to the OS-dependent back-end.
\end{itemize}

\section{Back-End}

\begin{itemize}
\item
Back-end receives parameter from front-end.
\item
Parameter is handled based on type (i.e. BANDWIDTH, DELAY, etc.)
\item
Back-end object calls OS-dependent routines to modify the desired parameter and make it take effect
\end{itemize}
\end{document}             % End of document.
