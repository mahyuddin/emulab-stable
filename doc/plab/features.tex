\section{Features}
\label{features}

\small{
\begin{verbatim}
Also describe it in terms of ``services'', from llp's 10/03 slide

His service taxonomy:
Slice == Experiment Portal
    Create expt/slice
        Resource Discovery [w/ Monitoring Service]
		Uses info from monitoring service
		Monitoring service:
			Node liveness, node resource availability (cpu/mem via loadavg, disk),
			e2e health, pair-wise paths.
        Resource Allocation
		Uses info from monitoring service
		Three ways, can be mixed
			fixed, site/node-centric (w/ types), link-centric [works?  no tunnels?]
		   port space (ssh; extend)
		Can mix with Emulab nodes (works?)

	*Admission control and optional queuing (queuing and retry service)

        Boot Slice [w/ Environment Service]

    Node/slice management (``environment service'')
	Node initialization
		Emulab state, per-user state
		Startup command

    Maintain expt/slice
        Software Upgrades [w/ Environment Service]
        Monitor Health [w/ Monitoring Service]
        Project membership

    *Control expt/slice
        Node 
        Expt

    *Naming service
	Virtual (DNS) or physical

    *Admin management service/interface
	Resource alloc params
	Admission control parameters
	Queue retry params


* = Not in llp's taxonomy; Elab-only

As of 10/03 Plab people were developing these; did not have anything:
   Environment Service
   Monitoring Service
   Resource allocation
\end{verbatim}
}%small



\subsection{Creating slices}

\subsubsection{Resource assignment}

Users of Emulab's PlanetLab interface have three choices for selecting the
nodes that will be used for their slice.

The most basic method is to manually chose nodes, as is done with the
PlanetLab's own current interface. This is accomplished by using Emulab's
'tb-fix-node' syntax. In addition to letting the user manually select nodes,
this method gives the user the opportunity to run their own selection algorithm
before submitting their experiment specification to Emulab.

Second, node selection can be done in a link-centric fashion. In this method,
the user specifies a set of virtual nodes, and a set of virtual links between
these nodes. Each virtual link can have a bandwidth, latency, and packet loss
rate specified for it. Emulab then matches, as best it can, these desired to
link characteristics to end-to-end characteristics observed between PlanetLab
nodes, gathered from third-party sensors. Emulab uses a custom mapper based on
a genetic algorithm to find a good match. If support for creating tunnels is
added to PlanetLab at some point in the future, we will optionally set up
tunnels to create an overlay network that matches the experimenter's virtual
topology.

The third, and most often used method of node selection is node-centric. In
this scheme, users ask for a set of virtual nodes, with no specific links
between them. Emulab's type system is used to select nodes with varying levels
of specificity - for example, a user can ask for a 'pcplab' node to get any
node in PlanetLab.  Or, they can ask for a 'pcplabinet' node to get a node on
the commodity Internet, or a 'pcplabdsl' node to get a node on a DSL line.
This is done using Emulab's 'tb-set-node-hardware' syntax.

This data is then fed to Emulab's resource mapper, assign. Assign uses
simulated annealing, a randomized heuristic algorithm to find a good mapping.
(See "A Solver for the Network Testbed Mapping Problem", Ricci, Lepreau,
Alfeld, SIGCOMM CCR April 2003.) Assign takes into account several mapping
goals, and can easily be expanded to take others into account as well.  At the
present time, it attempts to balance the following goals:

\begin{itemize}
\item Meeting type constraints mentioned above
\item Spreading slices out over the maximum possible number of distinct sites
\item Choosing nodes with low CPU and memory loads---Emulab can do admission
    control for experiments, if there are not enough planetlab nodes available
    with a specified CPU load or free memory
\end{itemize}

\subsection{Maintaining slices}

\subsection{Controlling slices}
