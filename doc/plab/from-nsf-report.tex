\item[Integration with and improved interface to PlanetLab.]
%% Note: doesn't directly correspond to any of our proposed milestones.
%%
%% Original author: Kirk Webb
%%
% Work on the Emulab portal to PlanetLab
%
In the interest of adding PlanetLab as an Emulab resource, we have put forth
significant development effort in bringing PlanetLab resources to Emulab
experimenters.  We have also developed a powerful, easy to use interface
targeted at native PlanetLab experimenters.

We provide a rich set of resource allocation semantics for experimenters,
including selection based on last-mile link type, load conditions, and
administrative/geographic domain (e.g., an experimenter can choose to allocate
a slice that spans one node at each PlanetLab ``site'').  Our sophisticated
resource mapping program (assign) uses these specifiable criterion to select
the set of nodes to use.

Underneath, Emulab is able to dynamically discover new PlanetLab nodes,
classify their last-mile link using an inference heuristic, and add them to its
database.  Once discovered, these nodes are immediately available for use by
experimenters.  Emulab also attempts to minimize the significant delays present
in PlanetLab vserver setup by running many instances of the initialization and
bootstrap processes in parallel.

To conserve resources and provide ongoing statistical feedback, and support
rich Experimental setup features, Emulab sets up a service vserver (sliver) on
every PlanetLab node it becomes aware of.  This service sliver reports node
load, provides liveness testing, and stores common vserver overlay files which
are utilized when a new Emulab-managed PlanetLab sliver is setup on a
particular node.

Over the past year, Emulab has tracked the RPC interface provided by PlanetLab,
which has been subject to abrupt change with no transition period.  Emulab's
PlanetLab back-end has been modularized for easy porting to new API schemas,
and has accounted for a wide range of resource and credential handling,
including resource tokens and per-slice authentication keys.

Additionally, due to the inherent instability in research platforms such as
PlanetLab, Emulab's back-end to it has been hardened to handle a variety of
failure modes (with retry).  Node outage, RPC call hangs, and other abnormal
termination and resource malformation (e.g., half setup vservers) have been
accounted for.  When we find that a PlanetLab node is problematic, we move it
out of production into a special pool where a back-end daemon at Emulab central
will test it.  This daemon will continually cycle through the nodes in the down
pool, testing them in batches.  When a node successfully runs through an end to
end PlanetLab vserver setup, it is released back into production.

\item[Planetlab EZ interface.]
%% Note: doesn't directly correspond to any of our proposed milestones.
%%
%% Original author: Rob Ricci
%%
One of the important advantages offered by \emulab's interface to PlanetLab
over PlanetLab's own native interface is an easy-to-use web interface for
configuring experiments (called ``slices'' in PlanetLab terminology).  In
contrast to the PlanetLab interface, which requires users to select individual
nodes manually, our interface also allows the option to ask for a certain
number of nodes, and \emulab automatically decides which ones to use.  Since a
substantial number PlanetLab nodes are frequently down, and some often see
heavy use, \emulab constantly monitors PlanetLab and only assigns nodes which
are up and have a reasonable amount of load.  The Web interface also provides a
mechanism for distributing files to every node in the experiment and for a
command to be run on each node at startup.  It provides an easy way to start an
experiment that uses every (currently up) node in PlanetLab, or one node at
each site.

\end{asparadesc}

%%%%%%%%%%%%%%%%%%%%%%%%%%%%%%%%%%%%%%%%%%%%%%%%%%%%%%%%%%%%%%%%%%%%%%%%%%%%%%%

%% End of file.
